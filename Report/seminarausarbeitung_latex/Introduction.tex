\chapter{Introduction}
\label{cha:introduction}

This semester's task in the HDL Lab is to design a general purpose processor implementing the ARM-Thumb2 architecture, and synthesize it with the Synopsis Design Compiler for the TSMC 45nm Standard Cell library. Further requirements are the RTL design with Verilog, the introduction of a pipelined design and the use of a given 1-Port memory. All these requirements have been met with our design, as pointed out in this report.\\ The paper is structured as following. At first, the different tasks are reviewed, and their distribution among the team members is outlined. An overview on our apporach is given. Then, the processor design and its submodules are reviewed. Furthermore, the RTL verification steps are shown. After a section on the synthesis process, the gate level verification results are shown. Finally, an evaluation of our design with regards to the task is done.

\section{Tasks and Work Distribution}
\label{sec:tasksandworkdistribution}

The task of the HDL Lab can be split into two major parts: micro-architecture design and rtl coding, and standard cell synthesis. Since the second part requires results of the first one, we started with the rtl design. 

First, the necessary submodules in the processor were detected, and each team member was assigned one or more modules to work on. Creating and running a testbench for each designed module must always be performed in time with the module itself, so we assigned that task to everyone designing a module. Along with the module design, requirements regarding the Thumb2-architecture had to be checked and built in. 

In the next step on the rtl level, finished and tested modules were integrated to verify their function. If necessary, a step back to the rtl design was taken in order to fix malfunctions in single modules or to adapt to changing requirements (e.g. changing communication between modules, new necessary signals).

The second major part, standard cell synthesis, was performed from the point where some modules were finished with the rtl design. Of course, for every change in the rtl design, the synthesis had to be re-done. With regard to the synthesis results for every module, some minor adaptions in the verilog code had to be done to guarantee the synthesizeability.

While the synthesis processes could only be started three days into the lab, they lasted until the final day. The RTL design also lasted until the end, since we tried to improve the design as far as possible. The approximate work distribution among the team members is shown below.

\begin{itemize}
\item \textbf{Lukas Boland:} RTL design (decoder) and top level verification
\item \textbf{Sven Str\"oher:} RTL design (ALU, Stack), synthesis, gate level verification
\item \textbf{Ana Carolina Ferreira:} RTL design (ALU, Fetch, controller), synthesis, gate level verification
\item \textbf{Julian K\"auser:} RTL design (memory interface, register file, controller), short report
\end{itemize}



