\chapter{Evaluation}
\label{cha:evaluation}

\section{Improvements}
\label{sec:improvements}
As we presented the micro-architecture and the design of our processor, there are yet possible improvements we encountered in the design phase. 

In the design, we particularly detected the following points for improvement:
\begin{itemize}
\item The \texttt{IT}-Instruction could have been implemented in the instruction fetch module more easily than in the decoder.
\item The utilized memory interface is the slowest version we have, due to problems with the alignment and the communication with the rest of the CPU. The faster interface would have accelerated the benchmark execution.
\item In our ALU, the \texttt{MULTIPLY}-operations are performed in a single cycle. This enlarges the ALU very much, so that a multi-cycle implementation would have decreased the ALU and also would have shortened the critical path.
\item In the current state of the design the instruction fetch is unstalled by the decoder unnecessarily late. This could be revised.
\end{itemize}

Regarding our design process in the group, we recognized some room for improvement, although this cannot be stated as clear as the technical details. In retrospect, we should have focused on making the benchmarks work on our processor first. Instead, we focused on the implementation of all instruction. This implies that we have to review the requirements more thoroughly before the start of work. With these information in hand, a more clear idea of what we had to design to make the processor work would have been present. Related to this observation is the fact that we had to revise the memory interface and throw away the hardware stack and the processor controller, so unnecessary work could have been avoided.
 
Most of the above mentioned improvements in the design or the design process  could not be included due to the tight time schedule. Nevertheless, especially the possible process changes could only be found out at a relatively late time in the lab or even afterwards, so that these have to be  applied in the next similar projects.

\newpage
\section{Conclusion}
\label{sec:conclusion}
Within the hardware design lab, a general-purpose processor for the Thumb instruction set has been designed by our group. It is working both on register transfer and on gate level. The processor can be operated at 793MHz clock frequency, which is mainly limited by the ALU. To speed up the execution, two pipeline stages have been included. 

Although the processor can execute the given benchmark applications and nearly the whole Thumb instruction set, there was no time for further improvements. Among the problems we faced were unclear requirements, synthesis problems (with both Verilog and the synthesis tools) and missing experience in project work. Thus, various steps had to be re-done, and the yet tight schedule for the lab was further limited. Ultimately, improvements were discarded in order to get an at least working processor. 
