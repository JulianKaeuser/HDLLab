\newpage
\thispagestyle{fancy}
\section{Introduction}
In this lab the design of a complex digital circuit, i.e. a THUMB processor with multiple pipeline stages, will be practised in teams of four students. System modelling will be performed in the hardware description language Verilog. Students will practise digital system design for a typical ASIC design flow.
Project work will be guided, but self-organised. i.e. given a range of proposals and best practises, groups determine responsibilities and organisation themselves.

\subsection{Goals}
\begin{itemize}
\item Use Verilog to model a processor that is defined on instruction set level.
\item Go through the digital design process encompassing specification, register-transfer-level modelling, simulation/verification, synthesis and gate-level simulation.
\item Evaluate the system's performance and resolve bottlenecks.
\item Practice group work, documentation and presentation techniques.
\item Apply course knowledge and use industry-grade tools, particularly:
	\begin{itemize}
	\item Modelsim  (Mentor Graphics)
	\item Design Compiler (Synopsys)
	\item bash, TCL, make
	\item GNU C toolchain for ARM
	\end{itemize}
\end{itemize}


\subsection{Expected Outcome}
This lab serves as preparation for a B.Sc./M.Sc. thesis in digital design at the Integrated Electronic Systems Lab. The practised methodology can directly be used as a template for a thesis and extended towards physical implementation targeting ASIC or FPGA.