\chapter{Tabellen}
\label{cha:tabellen}

Um eine einfache Tabelle in Latex zu erzeugen, kann man sich zun�chst mit der tabular � 
Umgebung begn�gen. Die Syntax daf�r ist die folgende: \\
 \\
\textbackslash begin \{tabular\}\{cols\} \\
Tabelleneintr�ge \\
\textbackslash end\{tabular\}\\
 \\
Im Argument ``cols'' gibt man f�r jede Spalte, die in der Tabelle stehen soll, die Ausrichtung 
an. Man hat dabei folgende Optionen: 
 


 
Will man die Spalten durch einen oder mehrere Striche trennen, so f�gt man mittels der 
Tastenkombination AltGr + ``<'' einen senkrechten Strich (|) zwischen (bzw. vor oder nach) 
den Ausrichtungsangaben ein (siehe Beispiele).  
Die Breite der Spalten bestimmt LATEX selbst, es sei denn man gibt eine Parbox an. Die 
Parbox hat den weiteren Vorteil, dass in einer Zelle der Tabelle mehrere Zeilen m�glich sind. 
Weiters kann man mittels Parboxen auch Tabulatoren wie im Word erzeugen. 
 
Die Tabelleneintr�ge werden zeilenweise eingegeben. Dabei erfolgt die Trennung innerhalb 
der Zeile mit ``\&''. Um in die n�chste Zeile zu springen hat man den Befehl ``\textbackslash \textbackslash''. Will man 
auch die Zeilen mittels Strichen trennen, so sind diese mit dem Befehl``\textbackslash hline'' an die 
entsprechende Stelle zu setzen.  
 
Das folgenden Beispiele erl�utern die tabular-Umgebung:  
Mit diesen Kommandos 

\textbackslash begin\{tabular\}\{1p\{10cm\}\}\\
\textbackslash textbf\{tabular\} \& Die Umgebung tabular wird in Latex gew�hlt um einfache Tabellen zu erzeugen\textbackslash\textbackslash\\
\textbackslash textbf\{table\} \& table wird gebraucht um abgesetzte Tabellen zu erzeugen. Die Tabelle kann damit als Gleitobjekt eingef�gt werden\textbackslash\textbackslash\\
\textbackslash textbf\{hline\} \& Mit diesem Befehl kann man mit einem Strich Zeilen trennen\textbackslash\textbackslash\\
\textbackslash textbf\{\textbackslash textbackslash\textbackslash textbackslash\} \& Um auf die n�chste Zeile zu springen\textbackslash\textbackslash\\
\textbackslash end \{tabular\}\textbackslash\textbackslash\\
\\
wird folgende Tabelle erzeugt:\\

\begin{tabular}{lp{10cm}}
\textbf{tabular} & Die Umgebung tabular wird in Latex gew�hlt um einfache Tabellen zu erzeugen\\
\textbf{table}& table wird gebraucht um abgesetzte Tabellen zu erzeugen. Die Tabelle kann damit als Gleitobjekt eingef�gt werden\\
\textbf{hline}& Mit diesem Befehl kann man mit einem Strich Zeilen trennen\\
\textbf{\textbackslash\textbackslash}& Um auf die n�chste Zeile zu springen\\
\end {tabular}\\
\\
\\
Dies ist ein Beispiel, wie man Parboxen zur ``Erzeugung'' von Tabulatoren benutzt werden 
kann. \\
\\
Tabellen, welche mittels der tabular-Umgebung geschrieben werden, werden einfach, wie ein 
gro�er Buchstabe in den Text eingef�gt. Um abgesetzte Tabellen zu erzeugen bedient man 
sich zus�tzlich der table-Umgebung, die die Tabelle als sogenanntes Gleitobjekt einf�gt. Sie 
hat folgende Syntax: 
 
\textbackslash begin\{table\}[Ausrichtung] \\
tabular-Umgebung \\
\textbackslash end\{table\}\\ 
\\
In die eckigen Klammern steht die Ausrichtung. Daf�r hat man folgende Optionen: 

\begin{table}[ht]
\begin{center}
\begin{tabular}{|l|p{8cm}|}
\hline
h&``here''- Tabelle soll an der selben Stelle wie im Quelltext eingerichtet werden\\
\hline
t&``top''- Tabelle wird an den unteren Rand der Seite gestellt\\
\hline
b&``bottom''- Tabelle wird an den unteren Rand der Seite gestellt\\
\hline
p&``page''- Tabelle wird auf einer Gleitobjektseite eingerichtet\\
\hline
\end{tabular}
\end{center}
\caption{table-Ausrichtung }
\label{tab: table Ausrichtung}
\end{table}%

\paragraph{Weiteres Beispiel}
 
Die folgenden Befehle erzeugen eine Tabelle mit Linien: 

\textbackslash begin\{table\}[!h]\\
\textbackslash begin\{center\}\\
\textbackslash begin\{tabular\}\{|l|r|r|r|r|r|\}\\
\textbackslash hline\&\$x\_\{min\}\$\&\$x\_\{0,25\}\$\&\$x\_\{0,5\}\$\&\$x\_\{0,75\}\$\&\$x\_\{max\}\$\textbackslash\textbackslash\\
\textbackslash hline
\textbackslash hline Alter \&14\&19\&20,5\&23\&70\textbackslash\textbackslash\\
\textbackslash hline Semester \&0\&2\&2\&4\&14\textbackslash\textbackslash\\
\textbackslash hline
\textbackslash end\{tabular\}\\


\begin{table}[ht]
\begin{center}
\begin{tabular}{|l|r|r|r|r|r|}
\hline&$x_{min}$&$x_{0,25}$&$x_{0,5}$&$x_{0,75}$&$x_{max}$\\
\hline
\hline Alter &14&19&20,5&23&70\\
\hline Semester &0&2&2&4&14\\
\hline
\end{tabular}
\end{center}
\caption{Verteilung von Alter und Semesteranzahl}
\end{table}


Mehr Informationen �ber Tabellen finden sie \htmladdnormallink{\underline{hier}}{http://www.ifas.jku.at/Portale/Institute/SOWI_Institute/ifas/content/e3413/e3414/files7134/TabelleninLATEX.pdf}.



