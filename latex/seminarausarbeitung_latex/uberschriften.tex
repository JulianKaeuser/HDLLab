\chapter{�berschriften}
\label{cha:uberschriften}

F�r solch eine Arbeit wird der Text in diverse Abschnitte, Unterabschnitte und evtl. auch noch Unterunterabschnitte unterteilt. Diese m�ssen dann mit der entsprechenden �berschrift gekennzeichnet werden. Die Dokumentklasse parkskip kennt hier die Unterscheidung in chapter, section, subsection, subsubsection und paragraph.

Der Befehl {\textbackslash chapter\{�berschrift1\}} beginnt ein neues Kapitel, erzeugt eine Kapitel�berschrift und tr�gt diese ins Inhaltsverzeichnis ein. So ist der oben genante Kapitel mit dem Befehl {\textbackslash chapter\{�berschriften\}} als chapter definiert.

Optional kann man mit Hilfe des Befehls {\textbackslash chapter[Kurzform]\{�berschrift\}} eine Kurzform f�r den Kapitelname angeben. Die Kurzform wird dann anstelle der �berschrift ins Inhaltsverzeichnis eingetragen.
\paragraph{Beispiel}
\label{par:beispiel}

{\textbackslash chapter[Anf�nge (1920)]\{Anf�nge der modernen Science--Fiction--Literatur (1920)\}}. Im Inhaltsverzeichnis erscheint nur "Anf�nge (1920)".

\section{�berschrift2}
\label{sec:uberschrift2}

Mit dem Befehl {\textbackslash section\{�berschrift2\}} wird einen neuen Abschnitt des Dokuments auf der section-Ebene erzeugt. Die zugeh�rige �berschrift wird definiert und ins Inhaltsverzeichnis eingetragen. Wenn eine Kurzform erw�nscht ist, kann sie auch mit dem entsprechenden Befehl erzeugt werden (Siehe Beispiel unter Kapitel\ref{par:beispiel}).

\subsection{�berschrift3}
\label{sec:uberschrift3}

In der subsection-Ebene, wird ebenfalls die zugeh�rige �berschrift erzeugt und ins Inhaltsverzeichnis eingetragen. Hier ist auch ebenfalls m�glich eine Kurzform angegeben. Der Befehl lautet folgenderma�en {\textbackslash subsection\{�berschrift3\}}.

\subsubsection{�berschrift4 und Paragraph}
\label{sec:uberschrift4}

Der Befehle dazu lauten {\textbackslash subsubsection\{�berschrift4\}} bzw. {\textbackslash paragraph\{"Paragraph\}}. Die �berschriften f�r diese Ebenen erfolgen ohne Nummerierung und werden nicht im Inhaltsverzeichnis aufgenommen. Das o.g. Beispiel wurde z. B. als Paragraph definiert.





